\chapter*{Introduction}

\setlength{\parindent}{4em}

\addcontentsline{toc}{chapter}{Introduction}

\section*{Pourquoi utiliser ce manuel?}

\noindent Le but de ce manuel de mathématique est d'introduire (ou de réintroduire) aussi rigoureusement que possible les concepts abordés au collège. Pour ce faire, il est nécessaire de voir d'autres concepts et d'expliquer certaines choses qui ne sont pas abordés lors d'un cursus standard. Par exemple, avant d'introduire les fonctions, nous étudierons ce qu'est un ensemble, afin de mieux pouvoir comprendre la construction d'une fonction. Cependant, ces concepts sont toujours expliqués de manière simple, sans pour autant enlever des détails important ou en simplifiant tellement que cela en devient faux. Lorsque quelque chose ne peut être expliqué rigoureusement (par exemple, la définition précise de ce qu'est un nombre réel), nous l'écrivons au lecteur et faisons notre mieux pour offrir une explication simple, tout en insistant sur le fait que c'est une simplification et qu'elle peut entraîner des questions auxquelles nous ne sommes pas en mesure de répondre compte tenu des connaissances du lecteur.

Vous constaterez aussi que l'organisation de ce manuel est peut-être un peu différente de celle d'un manuel scolaire standard. La raison derrière cela est simple: pourquoi écrire en vingt pages ce que nous pouvons faire en une? Dans un manuel normal, il n'est pas attendu de lire tout le contenu. Ici, étant donné la longueur du texte,

\setlength{\parindent}{0em}