\chapter{Notations}

\section{Système de notation numéraire}

Il existe de nombreuses façons d'écrire un nombre. Dans cette section, nous verrons plusieurs systèmes de notation particulièrement utilisés.

\subsection{Système décimal}

Le système décimal est le système de notation utilisé le plus couramment. Dans ce système, nous écrivons les nombres en utilisant les chiffres 0, 1, 2, 3, 4, 5, 6, 7, 8 et 9.

% TODO: Système hexadécimal
%\subsection{Système hexadécimal}

% TODO: Système binaire
%\subsection{Système binaire}

\subsection{Système romain}

Le système de numération romaine fut développé par les romains et provient de pratiques développées avant l'apparition de l'écriture---il y a donc très longtemps. Aujourd'hui, ils sont principalement utilisés pour désigner les siècles (ex.: \textsc{xxi}\textsuperscript{e} siècle) et le numéro d'ordre des noms de souverains (ex.: Louis XIV). Les règles d'utilisation de ce système d'écriture sont les suivantes:
\begin{itemize}
	\item 1 s'écrit \textsc{i}, 2 s'écrit \textsc{ii} et 3 s'écrit \textsc{iii};
	\item 5 s'écrit \textsc{v}, 10 s'écrit \textsc{x}, 50 s'écrit \textsc{l}, 100 s'écrit \textsc{c}, 500 s'écrit \textsc{d} et 1000 s'écrit \textsc{m};
	\item tout symbole qui suit un symbole de valeur supérieure ou égale s'ajoute à celui-ci (ex. : 6 s'écrit \textsc{vi});
	\item tout symbole qui précède un symbole de valeur supérieure se soustrait à ce dernier (ex.: 4 s'écrit \textsc{iv}).
\end{itemize}

Quelques exemples pour mieux comprendre: \textsc{xvi = x + v + i = 10 + 5 + 1 = 16}, \textsc{xl = l - x = 50 - 10 = 40}, \textsc{xiv = x + (v - i) = 10 + (5 - 1) = 14}.

\subsection{Notation scientifique}

Dans de nombreuses disciplines scientifiques telles que la physique (l'étude des phénomènes naturels de l'univers), la chimie (l'étude de certains types de transformations de la matière impliquant des éléctrons) ou la géologie (l'étude de la Terre), nous avons affaire à de très grands ou très petits nombres. Par exemple, un gogol (10 000 000 000 000 000 000 000 000 000 000 000 000 000 000 000 000 000 000 000 000 000 000 000 000 000 000 000 000 000 000 000 000 000) ou la masse d'un nucléon (0.000 000 000 000 000 000 000 167). La notation décimale devient alors très rapidement encombrante, surtout lors de calculs impliquant plusieurs de ces valeurs. Les mathématiciens ont donc eu l'idée d'utiliser les exposants (que nous avons vu dans le chapitre précédant) afin d'alléger la notation de ces très grands ou très petits nombres. Étant donné l'utilité de cette notation pour les matières scientifiques, ils nommèrent ce système de notation la \emph{notation scientifique}.

Voici quelques exemples de notations scientifique et de la notation décimale correspondante.

\begin{table}[H]
	\centering
	\begin{tabular}{|c|c|}
	\hline
	\textbf{Écriture décimale} & \textbf{Écriture scientifique} \\ \hline
	\begin{tabular}[c]{@{}c@{}}10 000 000 000 000 000 000 000 000 000 000 000 000 \\ 000 000 000 000 000 000 000 000 000 000 000 000\\  000 000 000 000 000 000 000 000 000\end{tabular} & $10^{100}$ \\ \hline
	0.000 000 000 000 000 000 000 167 & $1.67 \cdot 10^{-22}$ \\ \hline
	3 400 & $3.4 \cdot 10^3$ \\ \hline
	0.007 8 & $7.8 \cdot 10^{-3}$ \\ \hline
	\end{tabular}
\end{table}

%Le fonctionnement de la notation scientifique est comme suit. Nous commençons par noter le nombre de 0 qu'il y a soit à droite, soit à gauche de notre nombre (en fonction de là où ils se trouvent). Notons ce nombre $n$. Puis, nous prenons le reste du nombre et nous

\section{Préfixes}

Lorsque nous utilisons des nombres, nous souhaitons souvent quantifier une propriété du monde, tel qu'une distance, du temps ou de la masse. Pour ce faire, nous utilisons des unités telles que le mètre, la seconde ou le kilogramme. Cependant, tout comme dans le cas de la notation scientifique, il nous est utile de pouvoir écrire des grands et petits nombres plus facilement. Pour ce faire, nous pouvons modifier l'unité. Par exemple, lorsque nous mesurons une distance sur une feuille, nous utilisons des centimètres plutôt que des mètres. Voici un tableau de toutes les préfixes que nous pouvons apposer à une unité.

\begin{table}[H]
	\centering
	\begin{tabular}{|c|c|c|}
	\hline
	\textbf{Préfixe} & \textbf{Symbole} & \textbf{Forme exponentielle} \\ \hline
	exa & E & $10^{18}$ \\ \hline
	peta & P & $10^{15}$ \\ \hline
	tera & T & $10^{12}$ \\ \hline
	giga & G & $10^{9}$ \\ \hline
	mega & M & $10^{6}$ \\ \hline
	kilo & k & $10^{3}$ \\ \hline
	hecto & h & $10^{2}$ \\ \hline
	deca & da & $10^{1}$ \\ \hline
	--- & --- & $10^{0} = 1$ \\ \hline
	deci & d & $10^{-1}$ \\ \hline
	centi & c & $10^{-2}$ \\ \hline
	milli & m & $10^{-3}$ \\ \hline
	micro & $\mu$ & $10^{-6}$ \\ \hline
	nano & n & $10^{-9}$ \\ \hline
	pico & p & $10^{-12}$ \\ \hline
	femto & f & $10^{-15}$ \\ \hline
	atto & a & $10^{-18}$ \\ \hline
	\end{tabular}
\end{table}