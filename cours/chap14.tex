\chapter{Notations}

\section{Système de notation numéraire}

Il existe de nombreuses façons d'écrire un nombre. Dans cette section, nous verrons plusieurs systèmes de notation particulièrement utilisés.

\subsection{Système décimal}

Le système décimal est le système de notation utilisé le plus couramment. Dans ce système, nous écrivons les nombres en utilisant les chiffres 0, 1, 2, 3, 4, 5, 6, 7, 8 et 9.

% TODO: Système hexadécimal
%\subsection{Système hexadécimal}

% TODO: Système binaire
%\subsection{Système binaire}

\subsection{Système romain}

Le système de numération romaine fut développé par les romains et provient de pratiques développées avant l'apparition de l'écriture---il y a donc très longtemps. Aujourd'hui, ils sont principalement utilisés pour désigner les siècles (ex.: \textsc{xxi}\textsuperscript{e} siècle) et le numéro d'ordre des noms de souverains (ex.: Louis XIV). Les règles d'utilisation de ce système d'écriture sont les suivantes:
\begin{itemize}
	\item 1 s'écrit \textsc{i}, 2 s'écrit \textsc{ii} et 3 s'écrit \textsc{iii};
	\item 5 s'écrit \textsc{v}, 10 s'écrit \textsc{x}, 50 s'écrit \textsc{l}, 100 s'écrit \textsc{c}, 500 s'écrit \textsc{d} et 1000 s'écrit \textsc{m};
	\item tout symbole qui suit un symbole de valeur supérieure ou égale s'ajoute à celui-ci (ex. : 6 s'écrit \textsc{vi});
	\item tout symbole qui précède un symbole de valeur supérieure se soustrait à ce dernier (ex.: 4 s'écrit \textsc{iv}).
\end{itemize}

Quelques exemples pour mieux comprendre: \textsc{xvi = x + v + i = 10 + 5 + 1 = 16}, \textsc{xl = l - x = 50 - 10 = 40}, \textsc{xiv = x + (v - i) = 10 + (5 - 1) = 14}.

\subsection{Notation scientifique}

Dans de nombreuses disciplines scientifiques telles que la physique (l'étude des phénomènes naturels de l'univers), la chimie (l'étude de certains types de transformations de la matière impliquant des éléctrons) ou la géologie (l'étude de la Terre), nous avons affaire à de très grands ou très petits nombres. Par exemple, un gogol (10 000 000 000 000 000 000 000 000 000 000 000 000 000 000 000 000 000 000 000 000 000 000 000 000 000 000 000 000 000 000 000 000 000) ou la masse d'un nucléon (0.000 000 000 000 000 000 000 167). La notation décimale devient alors très rapidement encombrante, surtout lors de calculs impliquant plusieurs de ces valeurs. Les mathématiciens ont donc eu l'idée d'utiliser les exposants (que nous avons vu dans le chapitre~\ref{exposants}) afin d'alléger la notation de ces très grands ou très petits nombres. Étant donné l'utilité de cette notation pour les matières scientifiques, ils nommèrent ce système de notation la \textem{notation scientifique}.

Le fonctionnement de la notation scientifique est comme suit. Nous commençons par noter le nombre de 0 dont nous souhaitons

% TODO: Préfixes
% nano à giga
%\section{Préfixes}