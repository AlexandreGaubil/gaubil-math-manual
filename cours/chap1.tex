\chapter{Concepts de base}

La mathématique est un ensemble de concepts. Pour pouvoir comprendre la mathématique, il est important d'être familier avec certains concepts auparavant. Ces concepts nous permettrons de mieux comprendre les idées sous-jacentes de la mathématique---autrement dit, nous pourrons mieux comprendre pourquoi nous faisons certaines choses d'une certaine manière et non pas d'une autre.

\section{Logique}

La logique est l'étude de {\em relations entre proposition}. Une proposition est une idée qui peut être vraie, fausse, ou indéterminée (dont on ne sait pas si elle est vraie ou fausse). Par exemple, ``un triangle a trois sommets'' est une proposition qui est vraie, ``un chat a cinq pattes'' est une proposition qui est fausse et ``il fait soleil'' est une proposition indéterminée---nous ne savons pas à priori s'il fait soleil ou non. Nous pouvons établir des relations entre propositions. Par exemple, nous pouvons établir une relation d'implication (une chose entraîne ou implique une autre) entre les trois propositions suivantes: ``Minouchette est un chat'' et ``les chats ont quatre pattes'' implique ``Minouchette a quatre pattes''. En effet, si Minouchette est un chat et les chats ont quatre pattes, nous pouvons en déduire que Minouchette doit avoir quatre pattes.

Les connecteurs logiques (le terme formel pour relations) les plus communes sont les suivantes:
\begin{itemize}
    \item ``et'' ($\wedge$): ``Minouchette est un chat'' {\bf et} ``les chats ont quatre pattes'';
    \item ``ou'' ($\vee$): ``il fait jour'' {\bf ou} ``il fait nuit'';
    \item ``negation'' ($\neg$): la négation de ``Minouchette est un chat'' est ``Minouchette {\bf n}'est pas un chat'';
    \item ``implication'' ($\implies$): ``Minouchette est un chat'' et ``les chats ont quatre pattes'' {\bf implique} ``Minouchette a quatre pattes'';
    \item ``equivalence'' ($\iff$): ``il faut jour'' {\bf est équivalent à} ``il ne fait pas nuit''.
\end{itemize}


% TODO: add more
\begin{exercicefr}
    Indique les connecteurs logiques dans les propositions suivantes:
    \begin{exerciceenumnoeq}
        \item J'ai un chat et un chien.
        \item Je mangerai soit de la soupe, soit une salade.
        \item Si une figure a trois sommets, alors c'est un triangle.
    \end{exerciceenumnoeq}
\end{exercicefr}

\section{Axiomes}

Nous avons vu dans la section précédente que la logique est l'étude des relations entre propositions. En mathématique, nous prenons certaines propositions comme étant vraie (par exemple, nous considérons comme vrai le fait que 1 est un nombre entier). Nous appelons ces propositions que nous prenons comme vraie des {\em axiomes}. Il est important de noter que nous ne pouvons pas démontrer ces axiomes. Nous ne pouvons pas montrer que 1 est un nombre entier. Plutôt, nous le déclarons comme étant vrai. Pour certaines de ces vérités moins importantes, nous ne les appelons pas axiomes mais définitions. Ensuite, nous utilisons ces axiomes et definitions pour démontrer à l'aide d'une succession de relations entre propositions de nouvelles vérités que nous appelons théorèmes, lemmes, propositions, etc.

\begin{exemple}\label{concepts_de_base:axiomes_ex}
    Un autre axiome qui est utilisé en mathématique sont ``tout nombre entier a un successeur (un nombre qui vient après lui)''. Par exemple, nous notons le successeur de 1 $S(1)$ ou 2. De même, nous notons le successeur de 2 $S(2)$, $S(S(1))$ ou 3.
\end{exemple}

\section{Qualificatifs}

Dans l'exemple \ref{concepts_de_base:axiomes_ex}, nous avons vu que ``tout nombre entier a un successeur''. Il y a deux qualificatifs clés dans cet axiome: ``tout'' et ``a''. Le premier indique que la proposition s'applique à un ensemble de nombres