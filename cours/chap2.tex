\chapter{Notation}

\section{Système de notation numéraire}

Il existe de nombreuses façons d'écrire un nombre. Dans cette section, nous verrons plusieurs systèmes de notation particulièrement utilisés.

\subsection{Système décimal}

Le système décimal est le système de notation utilisé le plus couramment. Dans ce système, nous écrivons les nombres en utilisant les chiffres 0, 1, 2, 3, 4, 5, 6, 7, 8 et 9.

% TODO: Système hexadécimal
%\subsection{Système hexadécimal}

% TODO: Système binaire
%\subsection{Système binaire}

\subsection{Système romain}

Le système de numération romaine fut développé par les romains et provient de pratiques développées avant l'apparition de l'écriture---il y a donc très longtemps. Aujourd'hui, ils sont principalement utilisés pour désigner les siècles (ex.: \textsc{xxi}\textsuperscript{e} siècle) et le numéro d'ordre des noms de souverains (ex.: Louis XIV). Les règles d'utilisation de ce système d'écriture sont les suivantes:
\begin{itemize}
	\item 1 s'écrit \textsc{i}, 2 s'écrit \textsc{ii} et 3 s'écrit \textsc{iii};
	\item 5 s'écrit \textsc{v}, 10 s'écrit \textsc{x}, 50 s'écrit \textsc{l}, 100 s'écrit \textsc{c}, 500 s'écrit \textsc{d} et 1000 s'écrit \textsc{m};
	\item tout symbole qui suit un symbole de valeur supérieure ou égale s'ajoute à celui-ci (ex. : 6 s'écrit \textsc{vi});
	\item tout symbole qui précède un symbole de valeur supérieure se soustrait à ce dernier (ex.: 4 s'écrit \textsc{iv}).
\end{itemize}

Quelques exemples pour mieux comprendre: \textsc{xvi = x + v + i = 10 + 5 + 1 = 16}, \textsc{xl = l - x = 50 - 10 = 40}, \textsc{xiv = x + (v - i) = 10 + (5 - 1) = 14}.

\subsection{Notation scientifique}

Dans de nombreuses disciplines scientifiques telles que la physique (l'étude des phénomènes naturels de l'univers), la chimie (l'étude de certains types de transformations de la matière impliquant des éléctrons) ou la géologie (l'étude de la Terre), nous avons affaire à de très grands ou très petits nombres. Par exemple, un gogol (10 000 000 000 000 000 000 000 000 000 000 000 000 000 000 000 000 000 000 000 000 000 000 000 000 000 000 000 000 000 000 000 000 000) ou la masse d'un nucléon (0.000 000 000 000 000 000 000 167). La notation décimale devient alors très rapidement encombrante, surtout lors de calculs impliquant plusieurs de ces valeurs. Les mathématiciens ont donc eu l'idée d'utiliser les exposants (que nous avons vu dans le chapitre précédant) afin d'alléger la notation de ces très grands ou très petits nombres. Étant donné l'utilité de cette notation pour les matières scientifiques, ils nommèrent ce système de notation la \emph{notation scientifique}.

Voici quelques exemples de notations scientifique et de la notation décimale correspondante.

\begin{table}[H]
	\centering
	\begin{tabular}{|c|c|}
	\hline
	\textbf{Écriture décimale} & \textbf{Écriture scientifique} \\ \hline
	\begin{tabular}[c]{@{}c@{}}10 000 000 000 000 000 000 000 000 000 000 000 000 \\ 000 000 000 000 000 000 000 000 000 000 000 000\\  000 000 000 000 000 000 000 000 000\end{tabular} & $10^{100}$ \\ \hline
	0.000 000 000 000 000 000 000 167 & $1.67 \cdot 10^{-22}$ \\ \hline
	3 400 & $3.4 \cdot 10^3$ \\ \hline
	0.007 8 & $7.8 \cdot 10^{-3}$ \\ \hline
	\end{tabular}
\end{table}

%Le fonctionnement de la notation scientifique est comme suit. Nous commençons par noter le nombre de 0 qu'il y a soit à droite, soit à gauche de notre nombre (en fonction de là où ils se trouvent). Notons ce nombre $n$. Puis, nous prenons le reste du nombre et nous

\section{Préfixes}

Lorsque nous utilisons des nombres, nous souhaitons souvent quantifier une propriété du monde, tel qu'une distance, du temps ou de la masse. Pour ce faire, nous utilisons des unités telles que le mètre, la seconde ou le kilogramme. Cependant, tout comme dans le cas de la notation scientifique, il nous est utile de pouvoir écrire des grands et petits nombres plus facilement. Pour ce faire, nous pouvons modifier l'unité. Par exemple, lorsque nous mesurons une distance sur une feuille, nous utilisons des centimètres plutôt que des mètres. Voici un tableau de toutes les préfixes que nous pouvons apposer à une unité.

\begin{table}[H]
	\centering
	\begin{tabular}{|c|c|c|}
	\hline
	\textbf{Préfixe} & \textbf{Symbole} & \textbf{Forme exponentielle} \\ \hline
	exa & E & $10^{18}$ \\ \hline
	peta & P & $10^{15}$ \\ \hline
	tera & T & $10^{12}$ \\ \hline
	giga & G & $10^{9}$ \\ \hline
	mega & M & $10^{6}$ \\ \hline
	kilo & k & $10^{3}$ \\ \hline
	hecto & h & $10^{2}$ \\ \hline
	deca & da & $10^{1}$ \\ \hline
	--- & --- & $10^{0} = 1$ \\ \hline
	deci & d & $10^{-1}$ \\ \hline
	centi & c & $10^{-2}$ \\ \hline
	milli & m & $10^{-3}$ \\ \hline
	micro & $\mu$ & $10^{-6}$ \\ \hline
	nano & n & $10^{-9}$ \\ \hline
	pico & p & $10^{-12}$ \\ \hline
	femto & f & $10^{-15}$ \\ \hline
	atto & a & $10^{-18}$ \\ \hline
	\end{tabular}
\end{table}

\section{Language mathématique}

En mathématique, il y a plusieurs manières d'écrire une proposition. Un des but des cours de mathématique au collège et au lycée est d'apprendre une de ces manières qui fut conçut spécifiquement pour écrire des propositions mathématiques. Nous l'appelons le language mathématique. Considérons un exemple afin de mieux comprendre pourquoi nous utilisons ce language par opposition au français et quelles en sont les caractéristiques.

Considérons la proposition suivante, écrite en français: ``le nombre tel que, multiplié par trois et en ajoutant deux à ce résultat, est 8''. Bien que cette phrase soit compréhensible, elle prend beaucoup de place, est longue à lire, peut être la source d'incertitude, etc. Ceci est normal: le français (et tout autre language) ne fut pas créé pour écrire des propositions mathématiques et ne possède pas la rigueur nécessaire pour le faire correctement. Maintenant, réécrivons la proposition en language mathématique: ``$x : 3x+2 = 8$''. Nous savons immédiatement de quoi nous parlons: un nombre $x$. Puis, nous savons que nous ajoutons une condition (le ``$:$'' nous l'indique) qui spécifie de quel $x$ nous parlons. Cette condition prend la forme d'une équation: $3x+2 = 8$. Non seulement cette écriture est plus compacte, elle est aussi plus claire et nous permet de trouver la valeur de $x$ très rapidement en résolvant l'équation ($x = 2$).

\begin{exemple}
    Ces avantages ne sont pas nécessairement clair au niveau de mathématique que vous possédez. Considérons donc une proposition complexe---il est normal que vous ne la compreniez pas---mais qui indique encore plus clairement les avantage du language mathématique.

    \underline{Définition d'une contraction en français:} Soit un espace métrique avec une fonction de distance définie sur cet espace. Si pour une fonction allant de cet espace à lui-même il existe un nombre réel strictement compris entre zéro et un tel que la distance entre l'image d'un point par cette fonction et d'un deuxième point par cette fonction est inférieure à la distance entre ce point et ce deuxième point multiplié par ce nombre pour tous deux points appartenant à cet espace métrique, nous appelons cette fonction une contraction de cet espace métrique sur lui-même.

    \underline{Définition d'une contraction en language mathématique:} Soit $(X,d)$ un espace métrique. $(\phi \colon X \to X) \wedge (\exists c : 0<c<1) : d(\phi(x), \phi(y)) \leq c\cdot d(x,y) \Rightarrow \phi$ est une contraction de $X$ sur $X$.

    Il devient alors clair que le language mathématique est plus concis et plus clair.
\end{exemple}

Voici un tableau présentant divers éléments utilisés pour écrire des propositions en language mathématique.
\begin{table}[H]
    \centering
    \begin{tabular}{|c|c|c|}
        \hline
        \textbf{Symbole} & \textbf{Nom} & \textbf{Signification} \\ \hline
        $\exists \_\_\_$ & il existe & Il existe au moins un \_\_\_ \\ \hline
        $\exists ! \_\_\_$ & il existe un unique & Il existe exactement un seul \_\_\_ \\ \hline
        $\forall \_\_\_, ...$ & pour tout & Peut importe la valeur prise par \_\_\_, nous avons... \\ \hline
        $\_\_\_ : \_\_\_$ & tel que & \_\_\_ est définie de manière que \_\_\_ soit vrai \\ \hline
    \end{tabular}
\end{table}

\begin{exemple}
    Voici quelques exemples de propositions écrites en français et en language mathématique.
    \begin{enumerate}
        \item ``Pour tout nombre différent de zéro, il existe un inverse'' donne $\forall x : x\not=0, \exists y : y=\frac1x$.
        \item ``Il existe un nombre tel que ce nombre multiplié par 8, le tout moins 3, est égal à 10'' donne $\exists x : 8x - 3 = 10$.
    \end{enumerate}
\end{exemple}

\vspace{1em}

\begin{exercice}
    Écris les propositions suivantes en language mathématique.
    \begin{enumerate}
        \item Il existe un nombre tel que ce nombre moins deux soit égal à 9.
        \item Pour tout nombre, il existe un autre nombre tel que ce second nombre multiplié par deux égale le premier nombre.
    \end{enumerate}
\end{exercice}

\section{Conventions sur les inconnues}

\begin{prerequis}
	\textit{Arithmétique et Analyse.}
\end{prerequis}

Nous pouvons nommer nos variables et nos inconnues comme nous le souhaitons. Cependant, afin de rendre les démonstrations plus simples à relire, les mathématiciens suivent un certains nombre de conventions que nous allons présenter ci-dessous. S'il est possible de les respecter, il est préférable de le faire car ces conventions rendent l'argument plus rapide à lire et permet d'éviter des confusions possibles.

\subsection{Fonctions}

En général, une fonction générale est nommée $f$, $g$, ou $h$ dans cet ordre. Il vaut mieux éviter d'utiliser une apostrophe ou une seule lettre majuscule dans le nom d'une fonction (par exemple, $f'$ et $F$ sont à éviter) étant donné que ces notations sont utilisées pour représenter la dérivée et l'intégrale d'une fonction.

\subsection{Nombres réels}

Lorsque nous travaillons avec une équation, les noms d'inconnues sont généralement $x$, $y$ et $z$. On peut également utiliser $x_0$, $x_1$, $x_2$, etc. lorsqu'il s'agit d'une série de variables reliées entre elles par un point commun (par exemple, $h_0$ pour la hauteur du triangle 0, $h_1$ pour la hauteur du triangle 1, etc.).

\subsection{Alphabet grec}

L'alphabet grec est utilisé pour plein de choses en mathématique. Certaines lettres ont des valeurs particulières (par exemple, $\pi$), d'autres sont utilisées pour certaines types de valeurs (par exemple, $\epsilon$ pour de toutes petites valeurs positive) ou pour certaines formules (par exemple, $\Delta$ pour les équations du second degré). Il est donc important de connaître l'alphabet grec. Le tableau ci-dessous présente les lettres les plus utilisées et leur usage.

\begin{table}[H]
    \centering
    \begin{tabular}{|c|c|c|}
        \hline
        \textbf{Symbole} & \textbf{Nom} & \textbf{Usage} \\ \hline
        $\alpha$ & alpha & \\ \hline
        $\beta$ & beta & \\ \hline
        $\gamma$ ou $\Gamma$ & gamma & \\ \hline
        $\delta$ ou $\Delta$ & delta & $\delta$: une distance, $\Delta$: équations du second degré \\ \hline
		$\epsilon$ & epsilon & petite valeurs strictement positives ($\epsilon>0$) \\ \hline
		$\theta$ & theta & angles \\ \hline
		$\lambda$ & lambda & \\ \hline
		$\mu$ & mu & moyenne \\ \hline
		$\pi$ ou $\Pi$ & pi & \begin{tabular}[c]{@{}c@{}}$\pi$: rapport entre diamètre et aire d'un cercle\\ $\Pi$: représenter des produits\end{tabular} \\ \hline
		$\rho$ & rho & \\ \hline
		$\sigma$ ou $\Sigma$ & sigma & $\Sigma$: représenter des sommes \\ \hline
		$\phi$ & phi & fonctions \\ \hline
		$\chi$ & chi & \\ \hline
		$\omega$ & omega & \\ \hline
    \end{tabular}
\end{table}