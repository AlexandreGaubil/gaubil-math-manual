\documentclass[11pt]{article}
\usepackage{coursmath}

\begin{document}
\selectlanguage{french}
\begin{center}
	{\bf \LARGE \textcolor{red}{Programme}}
\end{center}

\section{Nombres et calculs}

\subsection{Utiliser les nombres pour comparer, calculer et résoudre des problèmes}

\begin{todolist}
    \item Utiliser diverses représentations d’un même nombre (écriture décimale ou fractionnaire, {\bf notation scientifique}, repérage sur une droite graduée) ; passer d’une représentation à une autre.
    \item[\done] Nombres décimaux.
    \item[\done] Nombres rationnels (positifs ou négatifs), notion d’opposé.
    \item[\done] Fractions, fractions irréductibles, cas particulier des fractions décimales.
    \item Définition de la racine carrée ; les carrés parfaits entre 1 et 144.
    \item Les préfixes de nano à giga.
    \item[\done] Comparer, ranger, encadrer des nombres rationnels. Repérer et placer un nombre rationnel sur une droite graduée.
    \item[\done] Ordre sur les nombres rationnels en écriture décimale ou fractionnaire.
    \item[\done] Égalité de fractions.
    \item[\done] Pratiquer le calcul exact ou approché, mental, à la main ou instrumenté. Calculer avec des nombres relatifs, des fractions ou des nombres décimaux (somme, différence, produit, quotient). Vérifier la vraisemblance d’un résultat, notamment en estimant son ordre de grandeur. Effectuer des calculs numériques simples impliquant des puissances, notamment en utilisant la notation scientifique.
    \item Définition des puissances d’un nombre (exposants entiers, positifs ou négatifs).
\end{todolist}

\subsection{Comprendre et utiliser les notions de divisibilité et de nombres premiers}

\begin{todolist}
    \item[\done] Déterminer si un entier est ou n’est pas multiple ou diviseur d’un autre entier. Simplifier une fraction donnée pour la rendre irréductible.
    \item Division euclidienne (quotient, reste).
    \item Multiples et diviseurs.
    \item[\done] Notion de nombres premiers.
\end{todolist}

\subsection{Utiliser le calcul littéral}

\begin{todolist}
    \item[\done] Mettre un problème en équation en vue de sa résolution. Développer et factoriser des expressions algébriques dans des cas très simples. Résoudre des équations ou des inéquations du premier degré.
    \item[\done] Notions de variable, d’inconnue.
    \item[\done] Utiliser le calcul littéral pour prouver un résultat
\end{todolist}

\section{Organisation et gestion de données, fonctions}

\subsection{Interpréter, représenter et traiter des données}

\begin{todolist}
    \item Recueillir des données, les organiser. Lire des données sous forme de données brutes, de tableau, de graphique.
    \item Calculer des effectifs, des fréquences.
    \item Tableaux, représentations graphiques (diagrammes en bâtons, diagrammes circulaires, histogrammes). Calculer et interpréter des caractéristiques de position ou de dispersion d’une série statistique.
    \item Indicateurs : moyenne, médiane, étendue.
\end{todolist}

\subsection{Comprendre et utiliser des notions élémentaires de probabilités}

\begin{todolist}
    \item Aborder les questions relatives au hasard à partir de problèmes simples. Calculer des probabilités dans des cas simples.
    \item Notion de probabilité.
    \item Quelques propriétés : la probabilité d’un événement est comprise entre 0 et 1 ; probabilité d’évènements certains, impossibles, incompatibles, contraires.
\end{todolist}

\subsection{Résoudre des problèmes de proportionnalité}

\begin{todolist}
    \item Reconnaître une situation de proportionnalité ou de non-proportionnalité.
    \item Résoudre des problèmes de recherche de quatrième proportionnelle.
    \item Résoudre des problèmes de pourcentage.
    \item Coefficient de proportionnalité.
\end{todolist}

\subsection{Comprendre et utiliser la notion de fonction}

\begin{todolist}
    \item Modéliser des phénomènes continus par une fonction. Résoudre des problèmes modélisés par des fonctions (équations, inéquations).
    \item Dépendance d’une grandeur mesurable en fonction d’une autre.
    \item Notion de variable mathématique.
    \item Notion de fonction, d’antécédent et d’image.
    \item Notations f(x) et x ? → f(x).
    \item Cas particulier d’une fonction linéaire, d’une fonction affine.
\end{todolist}

\section{Grandeurs et mesures}

\subsection{Calculer avec des grandeurs mesurables ; exprimer les résultats dans les unités adaptées}

\begin{todolist}
    \item Mener des calculs impliquant des grandeurs mesurables, notamment des grandeurs composées, en conservant les unités. Vérifier la cohérence des résultats du point de vue des unités.
    \item Notion de grandeur produit et de grandeur quotient.
    \item Formule donnant le volume d’une pyramide, d’un cylindre, d’un cône ou d’une boule.
\end{todolist}

\subsection{Comprendre l’effet de quelques transformations sur des grandeurs géométriques}

\begin{todolist}
    \item Comprendre l’effet d’un déplacement, d’un agrandissement ou d’une réduction sur les longueurs, les aires, les volumes ou les angles.
    \item Notion de dimension et rapport avec les unités de mesure (m, m2, m3).
\end{todolist}

\section{Espace et géométrie}

\subsection{Représenter l’espace}

\begin{todolist}
    \item (Se) repérer sur une droite graduée, dans le plan muni d’un repère orthogonal, dans un parallélépipède rectangle ou sur une sphère.
    \item Abscisse, ordonnée, altitude.
    \item Latitude, longitude. Utiliser, produire et mettre en relation des représentations de solides et de situations spatiales.
    \item Développer sa vision de l’espace.
\end{todolist}

\subsection{Utiliser les notions de géométrie plane pour démontrer}

\begin{todolist}
    \item Mettre en œuvre ou écrire un protocole de construction d’une figure géométrique.
    \item Coder une figure.
    \item Comprendre l’effet d’une translation, d’une symétrie (axiale et centrale), d’une rotation, d’une homothétie sur une figure.
    \item Résoudre des problèmes de géométrie plane, prouver un résultat général, valider ou réfuter une conjecture.
    \item Position relative de deux droites dans le plan.
    \item Caractérisation angulaire du parallélisme, angles alternes / internes.
    \item Médiatrice d’un segment.
    \item Triangle : somme des angles, inégalité triangulaire, cas d’égalité des triangles, triangles semblables, hauteurs, rapports trigonométriques dans le triangle rectangle (sinus, cosinus, tangente).
    \item Parallélogramme : propriétés relatives aux côtés et aux diagonales.
    \item Théorème de Thalès et réciproque.
    \item Théorème de Pythagore et réciproque.
\end{todolist}

\section{Algorithmique et programmation}

\begin{todolist}
    \item Décomposer un problème en sous-problèmes afin de structurer un programme ; reconnaître des schémas. Écrire, mettre au point (tester, corriger) et exécuter un programme en réponse à un problème donné.
    \item Écrire un programme dans lequel des actions sont déclenchées par des événements extérieurs.
    \item Programmer des scripts se déroulant en parallèle.
    \item Notions d’algorithme et de programme.
    \item Notion de variable informatique.
    \item Déclenchement d’une action par un évènement, séquences d’instructions, boucles, instructions conditionnelles.
\end{todolist}

\end{document}