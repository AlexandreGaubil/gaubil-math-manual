\section{Nombres relatifs}

Calculs automatisés: \url{https://calculatice.ac-lille.fr/spip.php?rubrique2}

\begin{exercice}
	Effectue les calculs suivants:
	\begin{exerciceenum}
		\item $5 - 11$
		\item $-3+9$
		\item $-8+5$
		\item $- 11 - 5$
		\item $7 - 12$
		\item $-4+2$
		\item $6+5$
		\item $- 6 - 7$
	\end{exerciceenum}
\end{exercice}

\begin{exercice}
	Effectue les calculs suivants:
	\begin{exerciceenum}
		\item $- 3.12 + 5.08 + 3.12$
		\item $6-5+4-3+2-1$
		\item $8-9+8-7+7-8+9-6$
		\item $2.3 - 1.8 + 3.7 - 1.2$
	\end{exerciceenum}
\end{exercice}

\begin{exercice}
	Effectue les calculs suivants:
	\begin{exerciceenum}
		\item $(3-2)-(2-7)$
		\item $3-(6-2)+(-5+2)$
		\item $4+3-2-(3+4)$
		\item $(-2+4)-(3-7)+3-5+1$
		\item $3 - (2 + 5) - (3 - 7) - 5$
		\item $1 - 4 + (3 + 5) - (2 + 7) + 2$
	\end{exerciceenum}
\end{exercice}

\begin{exercice}
	Effectue les calculs suivants:
	\begin{exerciceenum}
		\item $5 \times (-3)$
		\item $-9 \times 4$
		\item $-6 \times (-5)$
		\item $1.5 \times (-8)$
		\item $-6.3 \times (-7.8)$
		\item $- 3.9 \times 1.54$
		\item $6.1 \times (-5.6)$
	\end{exerciceenum}
\end{exercice}



\section{Équations}

\subsection*{Simplification d'écriture}

% Simplification d'écriture
\begin{exercice}
    Simplifie les expressions suivantes:
    \begin{exerciceenum}
        \item $7x+5x$
        \item $3y+12y$
        \item $12a-5a$
        \item $x+3x$
        \item $x+x^2+x+x+x$
        \item $4x(-5 + 1)$
        \item $-6(x - 5)$
        \item $(a+b)(a+b)$
        \item $2\times 3 \times a\times (b\times c)$
        \item $y^2 + x^3 + x^2 + x\times x + 4x$
    \end{exerciceenum}
\end{exercice}

\subsection*{Distributivité et factorisation}

% Distributivité simple - nombres
\begin{exercice}
    Distribue les calculs suivants, puis effectue les.
    \begin{exerciceenum}
        \item $8\times (17 + 8)$
        \item $5(20 - 7)$
        \item $-3(7-5)$
        \item $(9 - 6) \times (-8)$
        \item $(7 + 4) \times 3$
        \item $(11 + 9) \times (-2)$
    \end{exerciceenum}
\end{exercice}

% Distributivité simple - nombres et inconnues
\begin{exercice}
    Distribue les expressions suivantes.
    \begin{exerciceenum}
        \item $x(y + z)$
        \item $-x(-2+y)$
        \item $4(x - y)$
        \item $z(f + (-y))$
        \item $hj(e - t)$
        \item $(7 - d) h$
        \item $(4 + r) (-t)$
        \item $(- b+ 8) b$
    \end{exerciceenum}
\end{exercice}

\subsection{Tester une égalité}

% Calculer des valeurs
\begin{exercice}
    Calcule chacune des expressions suivantes pour $x=1$ et $y=4$.
    \begin{exerciceenum}
        \item $x^2+x+y$
        \item $x^2+2xy+y^2$
        \item $x^2y$
        \item $x^2+y^2$
    \end{exerciceenum}

    Calcule chacune des expressions suivantes pour $x=3$ et $y=2$.
    \begin{exerciceenum}
        \item $xy+4$
        \item $x-y+8$
        \item $xy-x-y+4$
        \item $xyx$
    \end{exerciceenum}
\end{exercice}

% Vérifier solutions
\begin{exercice}
    Vérifie si $x = 1$ et $y= 5$ sont des solutions des équations suivantes.
    \begin{exerciceenumnoeq}
        \item $x + y = 6$
        \item $2x - 4 = y - 7$
        \item $7 = y - (-2)$
        \item $8 = 2x + y$
    \end{exerciceenumnoeq}
\end{exercice}

\subsection{Résoudre une équation}

% Résoudre équation
\begin{exercice}
    Résout les équations suivantes.
    \begin{exerciceenumnoeq}
        \item $x+ 3 = 12$
        \item $\dfrac{x}{8} = 16$
        \item $-3x + 17 = 21 - x$
        \item $15 + 7x = 2x$
        \item $2x+8 = 6x$
        \item $3(5x + 9) = 4(7 + 5x)$
    \end{exerciceenumnoeq}
\end{exercice}