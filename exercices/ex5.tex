\section{Semaine du 11 Janvier 2021}

\begin{exercice}
    Résout les équations suivantes:
    \begin{exerciceenumnoeq}
        \item $5(6x - 9) + 3 = 8x - 4$
        \item $8x - 9 = 3(x - 1)$
        \item $(5(x - 7) + 2)(8x - 4) = 0$
        \item $\sqrt{25x^2} + 8 = -x$
    \end{exerciceenumnoeq}
\end{exercice}

\begin{exercice}
    Simplifie les écritures suivantes:
    \begin{exerciceenum}
        \item $a+4(b-a) + 7b^2(a^3 -b)$
        \item $(x^2 -4 + 5x)(x-1.5)$
        \item $\sqrt{25 x^2} \times \sqrt{81 y} + \sqrt{14}$
    \end{exerciceenum}
\end{exercice}

\begin{exercice}
    Un père a 27 ans et son fils en a 3. Dans combien d’années l’âge du fils sera-t-il égal au quart de celui de son père?
\end{exercice}

\begin{exercice}
    Christophe est chargé d’organiser l’excursion de sa classe. Il calcule le prix du voyage à 11€ par personne. Un élève devant renoncer à participer à l'excursion, le prix s’élève finalement à 11,50€. Combine d’élèves compte la classe de Christophe? Tu peux assumer que le prix total du voyage ne dépend pas du nombre d'étudiants qui viennent.
\end{exercice}

\begin{exercice}
    Le triple d’un nombre diminué de 12 est 108. Quel est ce nombre ?
\end{exercice}