\title{Devoir 1}
\author{Pour Émilie Gaubil}
\date{4 janvier 2021}

\maketitle

\exequationpremierdegree{2}

\exequationproduitnul{2}

\begin{exercice}
    Résout les équations suivantes:
    \begin{exerciceenumnoeq}
        \item $7(x - 4)= 5$
        \item $8(3x + 9) = -x$
    \end{exerciceenumnoeq}
\end{exercice}

\begin{exercice}
    Un père dispose de 1600~€ pour ses trois enfants. Il veut que l'aîné ait 200~€ de plus que le second et que le second ait 100~€ de plus que le dernier. Quelle somme doit il donner à chacun ?

    {\it Indice: pose le problème sous forme d'équation.}
\end{exercice}

\begin{exercice}
    Simplifie les écritures suivantes et, si possible, calcule les. Pour le \ref{racine}, écrit le sous la forme $\sqrt{a}$ ($a$ étant un nombre), puis calcule sa valeur. Attention! Certains calculs ne donne pas une solution exacte. Si c'est le cas, laisse les sous leur forme exact et donne une approximation du résultat (autrement dit, si une division donne une nombre avec une infinité de chiffres après la virgule, garde le résultat sous forme de fraction irréductible mais écrit tout de même le nombre avec un chiffre après la virgule).
    \begin{exerciceenum}
        \item $(a - b)(a+b^2)$
        \item $8x(9 - h + j)(j - j^3) + \sqrt{81}$
        \item \label{racine} $\sqrt{81} \times \sqrt{4}$
        \item $\sqrt{10 - 14}$
        \item $\frac{144}{208}$
        \item $\frac{4}{7} \times \frac{21}{30}$
    \end{exerciceenum}
\end{exercice}

\noindent Tu peux également regarder la vidéo ci-dessous si tu le souhaites, mais ce n'est pas obligatoire.

\noindent Introduction au théorème de Pythagore: \url{https://www.youtube.com/watch?v=QYM86GzWWG8}